\chapter{Introduction}
 This document lists C++ coding recommendations common in the C++ development community.
 The recommendations are based on established standards collected from a number of sources, individual experience, local requirements/needs, as well as suggestions.\newline
 There are several reasons for introducing a new guideline rather than just referring to the ones above. The main reason is that these guides are far too general in their scope and that more specific rules (especially naming rules) need to be established. Also, the present guide has an annotated form that makes it far easier to use during project code reviews than most other existing guidelines. In addition, programming recommendations generally tend to mix style issues with language technical issues in a somewhat confusing manner. The present document does not contain any C++ technical recommendations at all, but focuses mainly on programming style. For guidelines on C++ programming style refer to the C++ Programming Practice Guidelines.\newline
 While a given development environment (IDE) can improve the readability of code by access visibility, color coding, automatic formatting and so on, the programmer should never rely on such features. Source code should always be considered larger than the IDE it is developed within and should be written in a way that maximize its readability independent of any IDE.
 
 \section{Layout of the Recommendations}
 
 The recommendations are grouped by topic and each recommendation is numbered to make it easier to refer to during reviews.\newline
 Layout of the recommendations is as follows:\\[.5em]

\recommendation
{Guideline short description}
{Example if applicable}
{Motivation, background and additional information.}
\addtocounter{tabCounter}{-1}
\vspace{1cm}

\section{Recommendation Importance}
In the guideline sections the terms \textit{must}, \textit{should} and \textit{can} have special meaning. A must requirement must be followed, a should is a strong recommendation, and a can is a general guideline.

 