\section{Statements}
\subsection{Types}

\recommendation
{Types that are local to one file only can be declared inside that file.}
{}
{Enforces information hiding.}

\recommendation
{The parts of a class must be sorted public, protected and private. All sections must be identified explicitly. Not applicable sections should be left out.}
{}
{The ordering is \textit{"most public first"} so people who only wish to use the class can stop reading when they reach the protected/private sections.}

\recommendation
{Type conversions must always be done explicitly. Never rely on implicit type conversion.}
{floatValue = static\_cast\template{float}(intValue); // NOT: floatValue = intValue;}
{By this, the programmer indicates that he is aware of the different types involved and that the mix is intentional.}

\subsection{Variables}

\recommendation
{Variables should be initialized where they are declared.}
{}
{
	This ensures that variables are valid at any time. Sometimes it is impossible to initialize a variable to a valid value where it is declared:\newline
	int x, y, z;\newline
	getCenter(\&x, \&y, \&z);\newline
	In these cases it should be left uninitialized rather than initialized to some phony value.
}

\recommendation
{Variables must never have dual meaning.}
{}
{Enhance readability by ensuring all concepts are represented uniquely. Reduce chance of error by side effects.}

\recommendation
{Use of global variables should be minimized.}
{}
{In C++ there is no reason global variables need to be used at all. The same is true for global functions or file scope (static) variables.}

\recommendation
{Class variables should never be declared public (exception: performance bottle neck)}
{}
{
	The concept of C++ information hiding and encapsulation is violated by public variables. Use private variables and access functions instead. One exception to this rule is when the class is essentially a data structure, with no behavior (equivalent to a C struct). In this case it is appropriate to make the class' instance variables public.\newline
	Note that structs are kept in C++ for compatibility with C only, and avoiding them increases the readability of the code by reducing the number of constructs used. Use a class instead.
}

\recommendation
{C++ pointers and references should have their reference symbol next to the type rather than to the name.}
{
	float* x; // NOT: float *x;\newline
	int\& y;  // NOT: int \&y;
}
{The pointer-ness or reference-ness of a variable is a property of the type rather than the name. C-programmers often use the alternative approach, while in C++ it has become more common to follow this recommendation.}

\recommendation
{Implicit test for \textit{0} should not be used other than for boolean variables and pointers.}
{
	if (nLines != 0)  // NOT: if (nLines)\newline
	if (value != 0.0) // NOT: if (value)
}
{
	It is not necessarily defined by the C++ standard that ints and floats 0 are implemented as binary 0. Also, by using an explicit test the statement gives an immediate clue of the type being tested.\newline
	It is common also to suggest that pointers shouldn't test implicitly for 0 either, i.e. if (line == 0) instead of if (line). The latter is regarded so common in C/C++ however that it can be used.
}

\recommendation
{Variables should be declared in the smallest scope possible.}
{}
{Keeping the operations on a variable within a small scope, it is easier to control the effects and side effects of the variable.}

\subsection{Loops}


\begin{filecontents*}{\jobname.abc}
	isDone = false;
	while (!isDone) {
		//...
	}
\end{filecontents*}

\begin{filecontents*}{\jobname.def}
	//NOT:
	bool isDone = false;
	//...
	while (!isDone) {
		//...
	}
\end{filecontents*}

\recommendation
{Loop variables should be initialized immediately before the loop.}
{\lstinputlisting{\jobname.abc}
\lstinputlisting{\jobname.def}}
{}

\recommendation
{do-while loops can be avoided.}
{}
{
	\textit{do-while} loops are less readable than ordinary while loops and for loops since the conditional is at the bottom of the loop. The reader must scan the entire loop in order to understand the scope of the loop.\newline
	In addition, do-while loops are not needed. Any \textit{do-while} loop can easily be rewritten into a \textit{while} loop or a \textit{for} loop. Reducing the number of constructs used enhance readbility.
}


\begin{filecontents*}{\jobname.abc}
	while (true) {
		//...
	}

	for (;;) {  // NO!
		//...
	}

	while (1) { // NO!
		//...
	}
\end{filecontents*}

\recommendation
{The form while(true) should be used for infinite loops.}
{\lstinputlisting{\jobname.abc}}
{Testing against 1 is neither necessary nor meaningful. The form for (;;) is not very readable, and it is not apparent that this actually is an infinite loop.}

\subsection{Conditionals}


\begin{filecontents*}{\jobname.abc}
	bool isFinished = (elementNo < 0) || (elementNo > maxElement);
	bool isRepeatedEntry = elementNo == lastElement;
	if (isFinished || isRepeatedEntry) {
		//...
	}
	
	// NOT:
	if ((elementNo < 0) || (elementNo > maxElement)||
	elementNo == lastElement) {
		//...
	}
\end{filecontents*}

\recommendation
{Complex conditional expressions must be avoided. Introduce temporary boolean variables instead}
{\lstinputlisting{\jobname.abc}}
{By assigning boolean variables to expressions, the program gets automatic documentation. The construction will be easier to read, debug and maintain.}


\begin{filecontents*}{\jobname.abc}
	bool isOk = readFile (fileName);
	if (isOk) {
		//...
	}
	else {
		//...
	}
\end{filecontents*}

\recommendation
{The nominal case should be put in the if-part and the exception in the else-part of an if statement.}
{\lstinputlisting{\jobname.abc}}
{Makes sure that the exceptions don't obscure the normal path of execution. This is important for both the readability and performance.}


\begin{filecontents*}{\jobname.abc}
	if (isDone)  // NOT: if (isDone) doCleanup();
		doCleanup();
\end{filecontents*}

\recommendation
{The conditional should be put on a separate line.}
{\lstinputlisting{\jobname.abc}}
{This is for debugging purposes. When writing on a single line, it is not apparent whether the test is really true or not.}


\begin{filecontents*}{\jobname.abc}
	File* fileHandle = open(fileName, "w");
	if (!fileHandle) {
		//...
	}
	
	// NOT:
	if (!(fileHandle = open(fileName, "w"))) {
		//...
	}
\end{filecontents*}

\recommendation
{Executable statements in conditionals must be avoided.}
{\lstinputlisting{\jobname.abc}}
{Conditionals with executable statements are just very difficult to read. This is especially true for programmers new to C/C++.}

\subsection{Miscellaneous}

\recommendation
{The use of magic numbers in the code should be avoided. Numbers other than \textit{0} and \textit{1} should be considered declared as named constants instead.}
{}
{If the number does not have an obvious meaning by itself, the readability is enhanced by introducing a named constant instead. A different approach is to introduce a method from which the constant can be accessed.}


\begin{filecontents*}{\jobname.abc}
	double total = 0.0;    // NOT:  double total = 0;
	double speed = 3.0e8;  // NOT:  double speed = 3e8;
	
	double sum;
	//...
	sum = (a + b) * 10.0;
\end{filecontents*}

\recommendation
{Floating point constants should always be written with decimal point and at least one decimal.}
{\lstinputlisting{\jobname.abc}}
{
	This emphasizes the different nature of integer and floating point numbers. Mathematically the two model completely different and non-compatible concepts.\newline
	Also, as in the last example above, it emphasizes the type of the assigned variable (sum) at a point in the code where this might not be evident.
}

\recommendation
{Floating point constants should always be written with a digit before the decimal point.}
{double total = 0.5;  // NOT:  double total = .5;}
{The number and expression system in C++ is borrowed from mathematics and one should adhere to mathematical conventions for syntax wherever possible. Also, 0.5 is a lot more readable than .5; There is no way it can be mixed with the integer 5.}


\begin{filecontents*}{\jobname.abc}
	int getValue()   // NOT: getValue()
	{
		//...
	}
\end{filecontents*}

\recommendation
{Functions must always have the return value explicitly listed.}
{\lstinputlisting{\jobname.abc}}
{If not exlicitly listed, C++ implies int return value for functions. A programmer must never rely on this feature, since this might be confusing for programmers not aware of this artifact.}

\recommendation
{goto must not be used.}
{}
{Goto statements violate the idea of structured code. Only in some very few cases (for instance breaking out of deeply nested structures) should goto be considered, and only if the alternative structured counterpart is proven to be less readable.}

\recommendation
{\textit{0} should be used instead of \textit{NULL}.}
{}
{NULL is part of the standard C library, but is made obsolete in C++.}
